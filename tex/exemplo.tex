\documentclass[
  msc,proposal,
  oneside,
  hidecover,
  hidededication,
  hideack,
  hideepigraph,
  hidelof,
  hidelot,
  hideabstract,
  hidecover,
  extraporttitlepagefalse
]{ppgccufmg}    

\usepackage[english]{babel}
\usepackage[latin1]{inputenc}
\usepackage[T1]{fontenc}
\usepackage{type1ec}
\usepackage{graphicx}
\usepackage[a4paper,
  portuguese,
  bookmarks=true,
  bookmarksnumbered=true,
  linktocpage,
  colorlinks,
  citecolor=black,
  urlcolor=black,
  linkcolor=black,
  filecolor=black,
  ]{hyperref}
\usepackage[square]{natbib}

\begin{document}

\ppgccufmg{
  title={Web data extraction in semi-structured documents},
  authorrev={de Freitas Veneroso, Jo�o Mateus},
  cutter={D1234p},
  cdu={519.6*82.10},
  university={Federal University of Minas Gerais},
  course={Computer Science},
  address={Belo Horizonte},
  date={2017-11},
  advisor={Berthier Ribeiro de Ara�jo Neto},
  approval={img/approvalsheet.eps}
}

% ==========================================
% Beginning of text.                       |
% ==========================================

\chapter{Introduction}

Web data extraction is the task of automatically extracting structured information
from unstructured or semi-structured web documents. It is a subset of the broader
field of Information Extraction, and thus it faces many of the same challenges. 

Structured information such as that found in a well organized relational database must 
conform to an underlying data model, namely an abstract model that formalizes
the entities and relationships in a given application domain. Unstructured information
however is not organized according to a logical model and has useful information often 
permeated by large chunks of irrelevant data.

Tipically, Information Extraction tasks consist of mapping unstructured or poorly
structured information to a semantically well defined structure. The input is usually
composed of a set of documents that describe a group of entities in a similar manner 
while the Information Extraction task deals with identifying those entities and 
organizing them according to a template. 

As an example, consider a collection of 
novels and the task of identifying the name of the main character in each novel. 
For this task, the model must first identify proper nouns and then understand the 
text sufficiently to allow inference on the relative importance of each noun. 

To achieve such a goal it is often useful to employ methods developed in the disciplines 
of Information Retrieval and Natural Language Processing. The former has achieved a 
great deal of success in the task of classifying documents according to statistical 
properties and the latter led to huge improvements in modelling human language.
Many times, the various methods employed in these disciplines lead to different 
approaches in the field of Information Extraction.

In the scope of this work, we are interested in the semi-structured data usually
found in HTML web documents. Web documents most often lie in between the 
structured-unstructured data paradigm, meaning that they take a rather
relaxed approach in regard to formal structure. Hierarchy, element disposition,
class names, and other features related to document structure and indirectly associated
with the data itself are valuable information in the task of identifying entities and 
determining relationships. So much that many times they the major source of information 
for classification purposes such as when extracting information from standardized tables.
However, far from a structured database, web documents usually provide very limited structure
to otherwise unstructured data such as that found in free text. Take for example the
staff page for the intelligent robotics laboratory of Osaka University shown in figure 1.
Say we want to extract the name, position, email and picture of all members. It is easy to see
there is some sort of structure to the information we want to extract from this particular 
website, however, parts of the information are missing, repeated or disposed differently for
each particular member. If we want to go further it may be necessary to identify which
people are full members and which are visitors, a task that may pose a different challenge.



http://eng.irl.sys.es.osaka-u.ac.jp/home/member



It is often possible 










\chapter{Related Work}

Related work here.

\chapter{Methodology}

Methodology here.

\chapter{Schedule}

Schedule here.

% ==========================================
% End of text.                             |
% ==========================================

\ppgccbibliography{bibfile}
\nocite{*}

\end{document}
